\documentclass[a4paper,11pt]{article}
\usepackage{amssymb,enumerate}
\usepackage{amsmath}
\usepackage{bbm}
\usepackage{url}
\usepackage{cite}
\usepackage{graphics}
\usepackage{xspace}
\usepackage{epsfig}
\usepackage{subfigure}

\setlength\paperwidth  {210mm}%
\setlength\paperheight {300mm}%	

\textwidth 160mm%		% DEFAULT FOR LATEX209 IS a4
\textheight 230mm%

\voffset -1in
\topmargin   .05\paperheight	% FROM TOP OF PAGE TO TOP OF HEADING (0=1inch)
\headheight  .02\paperheight	% HEIGHT OF HEADING BOX.
\headsep     .03\paperheight	% VERT. SPACE BETWEEN HEAD AND TEXT.
\footskip    .07\paperheight	% FROM END OF TEX TO BASE OF FOOTER. (40pt)


\hoffset -1in				% TO ADJUST WITH PAPER:
	\oddsidemargin .13\paperwidth	% LEFT MARGIN FOR ODD PAGES (10)
	\evensidemargin .15\paperwidth	% LEFT MARGIN FOR EVEN PAGES (30)
	\marginparwidth .10\paperwidth	% TEXTWIDTH OF MARGINALNOTES
	\reversemarginpar		% BECAUSE OF TITLEPAGE.

%%%%%%%%%% Start TeXmacs macros
\newcommand{\tmtextit}[1]{{\itshape{#1}}}
\newcommand{\tmtexttt}[1]{{\ttfamily{#1}}}
\newenvironment{enumeratenumeric}{\begin{enumerate}[ 1.] }{\end{enumerate}}
\newcommand\sss{\mathchoice%
{\displaystyle}%
{\scriptstyle}%
{\scriptscriptstyle}%
{\scriptscriptstyle}%
}
\newcommand\PSn{\Phi_{n}}
\newcommand{\tmop}[1]{\ensuremath{\operatorname{#1}}}



\newcommand\Lum{{\cal L}}
\newcommand\matR{{\cal R}}
\newcommand\Kinnpo{{\bf \Phi}_{n+1}}
\newcommand\Kinn{{\bf \Phi}_n}
\newcommand\PSnpo{\Phi_{n+1}}
\newcommand\as{\alpha_{\sss\rm S}}
\newcommand\asotpi{\frac{\as}{2\pi}}

\newcommand\POWHEG{{\tt POWHEG}}
\newcommand\POWHEGBOX{{\tt POWHEG BOX}}
\newcommand\PYTHIA{{\tt PYTHIA}}
\newcommand\POWHEGpPYTHIA{{\tt POWHEG+PYTHIA}}
\newcommand\HERWIG{{\tt HERWIG}}

\def\lq{\left[} 
\def\rq{\right]} 
\def\rg{\right\}} 
\def\lg{\left\{} 
\def\({\left(} 
\def\){\right)} 

\def\beq{\begin{equation}}
\def\beqn{\begin{eqnarray}}
\def\eeq{\end{equation}}
\def\eeqn{\end{eqnarray}}

\def\mr{\mathrm}
\def\vbfevmv{VBF $e^+\nu_e\mu^-\bar\nu_\mu jj$\;}
\def\vbfww{VBF $W^+W^-jj$\;}
\def\wpm{W^+W^-}
\def\evmv{e^+\nu_e\mu^+\nu_\mu}
\def\pbox{{\tt POWHEG BOX}}
\def\pwg{{\tt POWHEG}}
\def\mc{\mathcal}

%%%%%%%%%% End TeXmacs macros

\title{Manual for $Zjj$ production via VBF in the \POWHEGBOX{}}
\date{}
%\author{}
%\keywords{}
%\abstract{}
%\preprint{}


\begin{document}
\maketitle
%
\noindent
The {\tt VBF\_Z} program is an implementation of the cross section for electroweak
$Z jj$ production in hadronic collisions within the \pbox{} framework. 
%
\\[2ex]
%
Special measures have been taken in order to avoid numerical instabilities due to singular configurations of the Born contributions (see Ref.~\cite{JSZ} for a detailed description). We recommend running the code with the flag {\tt withdamp} set to {\tt 1} and using a so-called Born-suppression factor that vanishes whenever a singular region of the Born phase space is approached. To this end, the flag {\tt bornsuppfact} should be set to {\tt 1} in the file {\tt powheg.input}. Collinear $\gamma^\star\to \ell^+\ell^-$ splittings are avoided by requiring a non-vanishing value of the invariant di-lepton mass $m_{\ell\ell}$. This value can be set by the user via the variable {\tt mll\_gencut} in the input file. We recommend a value larger than $10$~GeV. 
%
Please note that, because of these measures, the code is not expected to provide reliable results for observables analyzed with very inclusive cuts. 
%
\\[2ex]
%
This document describes the input parameters that are specific to the
implementation of $Zjj$ production via VBF. 
%
The parameters that are common to all {\tt POWHEG BOX} implementations are given in
the {\tt manual-BOX.pdf} document, in the {\tt POWHEG-BOX/Docs}
directory.
\\[2ex]
If you use this program, please quote
Refs.~\cite{JSZ,Alioli:2010xd}.


\section*{Running the program}
%
Download the {\tt POWHEG BOX}, following the instructions at the web site 
\\[2ex]
{\tt http://powhegbox.mib.infn.it/}
\\[2ex] 
and go to 
\\[2ex]
{\tt \$ cd POWHEG-BOX/VBF\_Z}
\\[2ex]
Running is most conveniently done in a separate directory, for instance do
\\[2ex]
{\tt \$ mkdir testrun}
\\[2ex]
The directory must contain the {\tt powheg.input} file and, for
parallel running, a {\tt pwgseeds.dat} file (see {\tt manual-BOX.pdf}
and {\tt Manyseeds.pdf}).
\\[2ex]
Before compiling make sure that:
\begin{itemize}
\item 
{\tt fastjet} is installed and {\tt fastjet-config} is in the path,
\item 
{\tt lhapdf} is installed and {\tt lhapdf-config} is in the path,
\item
{\tt gfortran}, {\tt ifort} or {\tt g77} is in the path, and the
appropriate libraries are in the environment variable {\tt
  LD\_LIBRARY\_PATH}. 
\end{itemize}
%
If {\tt LHAPDF} or {\tt fastjet} are not installed, the code can still
be run using a dummy analysis routine and built-in PDFs, see the {\tt
  Makefile} in {\tt VBF\_Z}.
%
\\[2ex]
After compiling, enter the testrun directory:
\\[2ex]
{\tt \$ cd testrun}
\\[2ex]
The program can be run sequentially by deactivating the variable {\tt manyseeds} in {\tt powheg.input},  or in parallel by setting {\tt manyseeds} to {\tt 1} in {\tt powheg.input}. In the latter case, when executing
\\[2ex]
{\tt \$../pwhg\_main}
\\[2ex]
the program will ask you to
\\[2ex]
{\tt enter which seed}
\\[2ex]
The program requires you to enter an index that specifies the line
number in the {\tt pwgseeds.dat} file where the seed of the random
number generator to be used for the run is stored. All results
generated by the run will be stored in files named {\tt
  *-[index].*}. When running on parallel CPUs, make sure that each
parallel run has a different index.
\\[2ex]
The program can be run in several steps. Each new step requires the
completion of the previous step.
\\[2ex]
The timings given in the following refer to the program compiled with
{\tt gfortran} and run on a cluster with 2.7~GHz Opteron processor.
% 
%%%%%%%%%%%%%%%%%%%
\subsection*{Step 1}
%%%%%%%%%%%%%%%%%%%
%
Consists of a single run to generate the grid. At this point the user
has to decide whether the $Z$~bosons are to be generated on-shell
({\tt zerowidth = 1}) or off-shell, distributed according to a
Breit-Wigner distribution ({\tt zerowidth = 0}). 
The user can also select the decay modes of
the weak bosons by assigning appropriate PDG code to the parameter
{\tt vdecaymode} in {\tt powheg.input}. Note
that only leptonic decays are supported. The template analysis file
{\tt pwhg\_analysis.f} needed in subsequent steps of the analysis is
designed for the $\ell^+\ell^-$ mode, where $\ell$ denotes an electron- or muon-type charged lepton.

We recommend to generate the grid with the option {\tt fakevirt 1} in
{\tt powheg.input}. When using this option, the virtual contribution
is replaced by a fake one proportional to the Born term. This speeds
up the generation of the grid.
\\[2ex]
One needs at least 3000000 events and 3 iterations. Set the following
tokens in the {\tt powheg.input} file:
\\[2ex]
{\tt ncall1 3000000
  \\[2ex]
  itmx1 3
  \\[2ex]
  ncall2 0
  \\[2ex]
  fakevirt 1 }
\\[2ex]
Run the program via
\\[2ex]
{\tt \$../pwhg\_main}
\\[2ex]
When prompted
\\[2ex]
{\tt enter which seed}
\\[2ex]
enter 1 or any other valid seed number.
\\[2ex]
This step takes roughly 28 hours of CPU. By setting {\tt ncall2 0} in
the {\tt powheg.input} file the program stops after the compilation of
this step.

%%%%%%%%%%%%%%%%%%%
\subsection*{Step 2}
%%%%%%%%%%%%%%%%%%%
%
Runs in parallel can be performed now. Comment out the {\tt fakevirt}
token from {\tt powheg.input}.
\\[2ex]
The runs must be performed in the directory where the previously generated grid is stored.
\\[2ex]
The integration and upper bound for the generation of btilde can be
performed with 50-100 runs with 5000-10000 calls each. Set for
instance
\\[2ex]
{\tt ncall2 5000}
\\[2ex]
{\tt itmx2 1}
\\[2ex]
in {\tt powheg.input}.
\\[2ex]
Folding numbers that are appropriate for runs at LHC energy are:
\\[2ex]
{\tt foldcsi 5 ! number of folds on csi integration}
\\[2ex]
{\tt foldy 5 ! number of folds on y integration}
\\[2ex]
{\tt foldphi 10 ! number of folds on phi integration}
\\[2ex]
Time is about 2 hours of CPU for each run with {\tt ncall2=5000}.
\\[2ex]
Setting
\\[2ex]
{\tt nubound 0}
\\[2ex]
in {\tt powheg.input} causes the program to stop after the completion
of this step.
\\[2ex]
In order to run, for example, 100 processes in parallel do:
\\[2ex]
{\tt \$../pwhg\_main}
\\[2ex]
When prompted
\\[2ex]
{\tt enter which seed}
\\[2ex]
enter an index for each run (from 1 to 100). The {\tt pwgseeds.dat}
must contain at least 100 lines, each with a different seed.
\\[2ex]
Upon the completion of this step, for each parallel run a file {\tt
  pwgNLO-*.top} is generated (where the * denotes the integer
identifier of the run).  These files contain the histograms defined in
{\tt pwhg\_analysis.f} at NLO-QCD accuracy, if the variable {\tt
  bornonly} is set to zero in {\tt powheg.input}.  Setting {\tt
  bornonly} to 1 yields the respective LO results. In either case, the
individual results of the parallel runs can be combined with the help
of the {\tt combineplots.f} file contained in the {\tt testrun}
directory.  To this end, just compile the file by typing, e.g.,
\\[2ex]
{\tt \$ gfortran combineplots.f}
\\[2ex]
and run the resulting executable. The program will ask for the 
number of {\tt pwgNLO-*.top} files in the directory and the number
of calls per run which are set by {\tt ncall2} in {\tt
  powheg.input}. Finally, enter the integer identifier of the first
file. The program will then generate the files {\tt combinedNLO.top}
and {\tt combinedNLO-gnu.top} which contain the combined histograms
in two different formats (topdrawer or gnuplot friendly). 

%%%%%%%%%%
\subsection*{Step 3}
%%%%%%%%%%
%
Also this step can be run in parallel. The number of processes can 
not be larger than the one used in the previous step. Setting
\\[2ex]
{\tt numevts 0}
\\[2ex]
{\tt nubound 100000}
\\[2ex]
takes roughly 2.5 hours per process.
\\[2ex]
The setting {\tt numevts 0} causes the program to stop after
completion of this step.  The parallel execution of the program is
performed as in the previous step.

%%%%%%%%%%%%
\subsection*{Step 4}
%%%%%%%%%%%%
%
Set {\tt numevts} to the number of events you want to generate per
process, for example
\\[2ex]
{\tt numevts 1000}
\\[2ex]
and run in parallel. Again, the number of processes must not be larger than
the one used in the previous step. Generating the specified number of
events takes about 65 hours per process.
\\[2ex]
At this point, files of the form {\tt pwgevents-[index].lhe} are
present in the run directory.
\\[2ex]
Count the events:
\\[2ex]
{\tt \$ grep '/event' pwgevents-*.lhe | wc}
\\[2ex]
The events can be merged into a single event file by
\\[2ex]
{\tt cat pwgevents-*.lhe | grep -v '/LesHouchesEvents' >
  pwgevents.lhe}


%%%%%%%%%%%%%%%%%%
\section*{Analyzing the events}
%%%%%%%%%%%%%%%%%%
%
It is straightforward to feed the {\tt *.lhe} events into a generic
shower Monte Carlo program, within the analysis framework of each
experiment. We also provide a sample analysis that computes several
histograms and stores them in topdrawer output files.
\\[2ex]
Doing (from the {\tt VBF\_Z} directory)
\\[2ex]
{\tt \$ make lhef\_analysis}
\\[2ex]
{\tt \$ cd testrun}
\\[2ex]
{\tt ../lhef\_analysis}
\\[2ex]
analyzes the bare {\tt POWHEG BOX} output, creating the topdrawer file
{\tt LHEF\_analysis.top}. The targets {\tt main-HERWIG-lhef} and {\tt
  main-PYTHIA-lhef} are instead used to perform the analysis on events
fully showered using {\tt HERWIG} or {\tt PYTHIA}. Various setting of
the respective parton shower Monte Carlo can be modified by editing the files {\tt
  setup-PYTHIA-lhef.f} and {\tt setup-HERWIG-lhef.f} respectively.

%%%%%%%%%%%%%%%%%%%%%%%%%%
%
\begin{thebibliography}{99}

\bibitem{JSZ} B.~J\"ager, S.~Schneider, G.~Zanderighi, {\em
    Next-to-leading order QCD corrections to electroweak $Zjj$ 
    production in the POWHEG BOX}, arXiv:1207.2626 [hep-ph].
  
\bibitem{Alioli:2010xd} S.~Alioli, P.~Nason, C.~Oleari and E. Re, {\em
    A general framework for implementing NLO calculations in shower
    Monte Carlo programs: the POWHEG BOX}, JHEP {\bf 1006} (2010)
  043  [arXiv:1002.2581 [hep-ph]].

\end{thebibliography}
%%%%%%%%%%%%%%%%%%
\end{document}
